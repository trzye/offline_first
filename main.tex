\documentclass[a4paper, 12pt]{article}
\usepackage[T1]{fontenc}
\usepackage{polski}
\usepackage[utf8]{inputenc}
\usepackage[polish]{babel}
\usepackage[margin=1in]{geometry}
\usepackage{graphicx}
\usepackage{wrapfig}
\usepackage{fancyhdr}
\usepackage{lastpage}
\usepackage[ddmmyyyy]{datetime}
\renewcommand{\dateseparator}{/}
\fancyhead{}
\usepackage{hyperref}
\renewcommand{\headrulewidth}{0pt}

\pagestyle{fancy}
\cfoot{\thepage\hspace{1pt}/\pageref{LastPage}}


\begin{document}


\begin{wrapfigure}{L}{20px}
\includegraphics[width=1.5cm,height=1.3cm,keepaspectratio]{ee.png}
\end{wrapfigure}

Politechnika Warszawska 
\hfill 11/01/2017

Wydział Elektryczny


\quad
\begin{center}
\center \Huge Aplikacje typu Offline-First
\vspace{0.5cm}\\
\small Jereczek Michał
\end{center}

\tableofcontents
\pagebreak

\section{Czym są aplikacje Offline-First?}
W roku 2017 nie trzeba już nikogo przekonywać, że urządzenia mobilne przejęły rynek. Smartfony i tablety pozwalają nam (programistom) na zdecydowanie większe możliwości przy tworzeniu aplikacji, jednak projektując aplikacje mobilne często zapominamy o dużym ograniczeniu względem aplikacji desktopowych - jakości połączania internetowego.

Offline-First jest ruchem i ideologią mającą na celu przypomnieć nam o tym, że urządzenia mobilne przez swoją przenośną istotę są bardzo podatne na utratę połączenia internetowego. Do najważniejszych czynników, które mogą doprowadzić do utraty połączenia są:
\begin{itemize}
\item \textbf{Utrata zasięgu sieci komórkowej} - czynników może być wiele: awaria stacji bazowej, w której zasięgu jest telefon, wjazd do tunelu, przemieszczanie się metrem, zakłócenia sygnałów.
\item \textbf{Utrata środków na koncie} - zablokowanie internetu przez operatora sieci z powodu niewystarczających środków.
\item \textbf{Utrata połączenia WiFi} - poprzez wyjście poza zasięg routera bądź jego awarię.
\end{itemize}
Najbardziej odczuwalna jest dynamiczność związana z poruszaniem się w obszarach miejskich, telefon co chwila przełącza się między stacjami bazowymi, traci zasięg na rzecz wjazdu do tunelu czy też podróży metrem. 

Naszym zadaniem (zadaniem programistów) jest uwzględnienie tych czynników podczas projektowania aplikacji mobilnej. Bardzo często zapominamy o tym przyzwyczajeni do dobrych warunków w jakich testujemy nasze dzieła.

\section{Najczęściej spotykane błędy}
Od czego zacząć przy projektowaniu aplikacji mobilnej? Najlepiej od przypomnienia sobie wszystkich momentów, w których utrata sieci w połączeniu z słabo zaprojektowaną aplikacją zirytowały nas jako jej użytkowników. Do najczęstszych błędów należą:
\begin{itemize}
\item \textbf{Przeprowadzenie przez całą funkcjonalność aplikacji ostatecznie nie pozwalając jej sfinalizować.} 

Przykładem tutaj może być aplikacja do dzielenia się zdjęciami. Użytkownik nieświadomy braku połączenia dodaje nowe zdjęcie i poświęca kilka minut na jego cudowny opis - w momencie przyciśnięcia magicznego przycisku ,,wyślij'' okazuje się jednak, że nie ma połączenia z internetem. Z takiej sytuacji wyszedł bardzo dobrze Snapchat, jeżeli zaistnieje taka sytuacja to tak zwany snap zostanie zapisany w pamięci aplikacji i wysłany przy najbliższej możliwej okazji.
\item \textbf{Całkowite zablokowanie funkcji aplikacji bez połączenia z internetem. }

Warto zastanowić się czy wszystko, co udostępnia nasza aplikacja koniecznie musi być poprzedzone przymusowym logowaniem (popularnym teraz jedno-klikiem z Facebook'a czy też innego portalu społecznościowego). Dobre rozwiązanie stosuje tu chociażby Google Music, które pozwala nam na zapisywanie muzyki w pamięci urządzenia, wtedy - jeżeli nie posiadamy połączenia z internetem, możemy uruchomić aplikację w trybie offline i wszystkie wyniki zostaną zawężone do tych z pamięci telefonu.
\end{itemize}
\section{Podejście do projektowania}
O czym warto pamiętać podczas projektowania aplikacji typu Offline-First? Pierwszym ważnym krokiem jest podzielenie funkcjonalności na przykładowe grupy:
\begin{itemize}
\item \textbf{Funkcjonalności Online} - te funkcjonalności, które bezwzględnie wymagają połączenia z internetem.
\item \textbf{Funkcjonalności Online/Offline} - funkcjonalności, które do pełnego rezultatu wymagają połączenia, lecz mogą częściowo działać bez niego.
\item \textbf{Funkcjonalności Offline} - pozostałe funkcjonalności, które w ogóle nie potrzebują połączenia z internetem.
\end{itemize}
Po takiej klasyfikacji należy zastanowić się co można z nimi zrobić. Najważniejszym zadaniem jest przede wszystkim poinformowanie użytkownika, że nie ma połączenia z internetem. Jeżeli użytkownik wszedł właśnie do menu funkcjonalności związanej z połączeniem, to stan połączenia powinien być na bieżąco monitorowany, a użytkownik powinien albo dostać ostrzeżenie, albo widzieć w tle jasną informacją o braku połączenia, a co za tym idzie - ograniczonych możliwościach.

Dobrą praktyką jest również monitorowanie stanu połączenia i po prostu blokowanie dodatkowych opcji z poziomu menu użytkownika (np. wyszarzenie opcji odpowiedzialnej za udostępnianie jakiejś treści online).

Kolejnym ciekawym rozwiązaniem jest przetrzymywanie danych z poprzednich sesji aplikacji, aby pokazać je użytkownikowi gdy nie będzie miał połączenia. Dobrym przykładem jest tutaj aplikacja mobilna Facebooka, która zapisuje w pamięci poprzednią zawartość aktualności i pokazuje je gdy użytkownik nie będzie wstanie ich zaktualizować (oczywiście użytkownik jest informowany, że wyświetlane aktualności nie są aktualne). Dzięki temu użytkownik może z nudów poczytać mniej aktualne informacje (może znajdzie coś co wcześniej przeoczył?).

\section{Podsumowanie}
Podsumowując. Najważniejszym w aplikacjach Offline-First jest przede wszystkim to abyśmy mieli świadomość o ograniczeniach związanych z połączeniem. Jest to ważny czynnik, o którym trzeba pamiętać już na etapie projektowania aplikacji.

Aktualnie na rynku wygrywają aplikacje, które dostarczają najlepszych wrażeń użytkowania - pamiętając o Offline-First nasz produkt jest znacznie atrakcyjniejszy. \vspace{1cm}\\

\begin{thebibliography}{1}

\bibitem{DATA} {Alex Feyerke, ,,Designing Offline-First Web Apps'' [online][dostęp: 11.01.2017 \url{http://alistapart.com/article/offline-first}}

\end{thebibliography}

\end{document}